\documentclass[a4paper,12pt]{article}
\usepackage[utf8]{inputenc}
\usepackage[T1]{fontenc}
\usepackage{geometry}
\geometry{margin=1in}
\usepackage{setspace}
\usepackage{parskip}
\usepackage{enumitem}
\usepackage{titlesec}
\titleformat*{\section}{\large\bfseries\MakeUppercase}
\usepackage{noto}
\begin{document}
\begin{center}
{\large \textbf{LOAN AGREEMENT}} \\
\textit{Kasunduan sa Pautang} \\
\textit{Republic of the Philippines / Republika ng Pilipinas}
\end{center}
\vspace{0.5cm}
This Loan Agreement (``Agreement'') is made and entered into this 17th day of September, 2025, by and between:
\textit{(Ang Kasunduang ito (``Kasunduan'') ay ginawa at pinasok ngayong ika-17 ng September, 2025, sa pagitan ng:)}
\vspace{0.5cm}
\textbf{LENDER:} [Lender Name], a corporation duly organized and existing under the laws of the Republic of the Philippines, with principal office at [Lender Address] (hereinafter referred to as the ``Lender''); \\
\textit{(Tagapagpautang: [Lender Name], isang korporasyon na alinsunod sa mga batas ng Republika ng Pilipinas, na may pangunahing tanggapan sa [Lender Address] (na tatawaging ``Tagapagpautang''))}
\vspace{0.3cm}
\textbf{AND / At}
\vspace{0.3cm}
\textbf{BORROWER:} Slate Transportation Vehicle, a corporation duly organized and existing under the laws of the Republic of the Philippines, with principal office at 123 Rotonda Street, Quezon City (hereinafter referred to as the ``Borrower''). \\
\textit{(Nanghihiram: Slate Transportation Vehicle, isang korporasyon na alinsunod sa mga batas ng Republika ng Pilipinas, na may pangunahing tanggapan sa 123 Rotonda Street, Quezon City (na tatawaging ``Nanghihiram''))}
\vspace{0.5cm}
The Lender and Borrower may hereinafter be collectively referred to as the ``Parties'' and individually as a ``Party.'' \\
\textit{(Ang Tagapagpautang at Nanghihiram ay maaaring tawagin na ``Mga Panig'' at paisa-isa bilang ``Panig'')}
\section*{Article I – Loan Amount and Purpose}
1.1 The Lender agrees to extend to the Borrower a loan in the principal amount of PHP 1,200,000.00. (the ``Loan''). \\
\textit{(1.1 Sumasang-ayon ang Tagapagpautang na magpahiram sa Nanghihiram ng halagang PHP 1,200,000.00 (``Pautang''))}
1.2 The Loan shall be used exclusively for the purpose of Replacing Old Vehicles. The Borrower shall not divert the proceeds to any unlawful or unauthorized purpose. \\
\textit{(1.2 Ang Pautang ay gagamitin lamang para sa layunin ng Replacing Old Vehicles. Ipinagbabawal ang paggamit nito sa anumang ilegal o hindi pinahihintulutang layunin.)}
\section*{Article II – Disbursement}
2.1 The Loan proceeds shall be disbursed on or before September 17, 2025 via Cash, subject to the Borrower’s compliance with all conditions precedent. \\
\textit{(2.1 Ang halaga ng Pautang ay ibibigay sa o bago ang September 17, 2025 sa pamamagitan ng Cash, kung natugunan ng Nanghihiram ang lahat ng kinakailangan.)}
2.2 The Lender reserves the right to withhold disbursement if the Borrower fails to submit required documents or collateral. \\
\textit{(2.2 May karapatan ang Tagapagpautang na ipagpaliban ang paglabas ng Pautang kung hindi naibigay ng Nanghihiram ang mga kinakailangang dokumento o kolateral.)}
\section*{Article III – Interest and Charges}
3.1 The Loan shall bear interest at the rate of 10\% per annum, computed on the outstanding balance. \\
\textit{(3.1 Ang Pautang ay magkakaroon ng interes sa rate na 10\% kada taon, na kinakalkula batay sa natitirang balanse.)}
3.2 Interest shall accrue from the date of disbursement and be payable in accordance with the repayment schedule. \\
\textit{(3.2 Ang interes ay magsisimula mula sa petsa ng pagbibigay at babayaran ayon sa iskedyul ng pagbabayad.)}
3.3 The Borrower shall be responsible for all ancillary charges, including documentary stamp tax, bank charges, and legal fees, if any. \\
\textit{(3.3 Ang Nanghihiram ang mananagot sa lahat ng karagdagang bayarin, kabilang ang documentary stamp tax, bayarin sa bangko, at bayarin sa legal, kung mayroon.)}
\section*{Article IV – Repayment}
4.1 The Borrower shall repay the Loan in 12 equal installments of PHP 105,499.06, due on the 25 day of each month, commencing on October 25, 2025. \\
\textit{(4.1 Ang Nanghihiram ay magbabayad ng Pautang sa 12 na pantay na hulugan ng PHP 105,499.06, na dapat bayaran sa ika-25 araw ng bawat buwan, simula sa October 25, 2025.)}
4.2 Payments shall be applied first to accrued interest, then to principal, and finally to penalties or other charges. \\
\textit{(4.2 Ang mga bayarin ay uunahin sa naipon na interes, pagkatapos sa prinsipal, at panghuli sa mga parusa o iba pang bayarin.)}
\section*{Article V – Security}
5.1 The Loan shall be secured by the following collateral: AR. \\
\textit{(5.1 Ang Pautang ay sisiguruhin ng sumusunod na kolateral: AR.)}
5.2 The Borrower authorizes the Lender to enforce the security in case of default, without need of prior demand or judicial action, subject to applicable Philippine laws. \\
\textit{(5.2 Pinahintulutan ng Nanghihiram ang Tagapagpautang na ipatupad ang kolateral sa kaso ng default, nang walang pangangailangan ng paunang kahilingan o aksyong hudisyal, alinsunod sa naaangkop na batas ng Pilipinas.)}
\section*{Article VI – Representations and Warranties}
The Borrower hereby represents and warrants that:
\textit{Ang Nanghihiram ay nagpapahayag at ginagarantiyahan na:}
\begin{itemize}
\item It has full power, legal capacity, and authority to enter into and perform its obligations under this Agreement.
\item \textit{Ito ay may buong kapangyarihan, legal na kapasidad, at awtoridad na pumasok at tuparin ang mga obligasyon nito sa ilalim ng Kasunduang ito.}
\item The execution of this Agreement has been duly authorized.
\item \textit{Ang pagpapatupad ng Kasunduang ito ay naaayon na awtorisado.}
\item All information and documents provided are true, accurate, and complete.
\item \textit{Lahat ng impormasyon at dokumentong ibinigay ay totoo, tumpak, at kumpleto.}
\end{itemize}
\section*{Article VII – Events of Default}
The Borrower shall be deemed in default upon occurrence of any of the following:
\textit{Ang Nanghihiram ay ituturing na nasa default kapag nangyari ang alinman sa mga sumusunod:}
\begin{itemize}
\item Failure to pay any installment when due.
\item \textit{Pagkabigo sa pagbabayad ng anumang hulugan kapag due na.}
\item Breach of any representation, warranty, or obligation herein.
\item \textit{Paglabag sa anumang representasyon, garantiya, o obligasyon dito.}
\end{itemize}
Upon default, the Lender may, at its sole discretion:
\textit{Sa kaso ng default, ang Tagapagpautang ay maaaring, sa kanyang sariling pagpapasya:}
\begin{itemize}
\item Declare the entire outstanding Loan immediately due and payable.
\item \textit{Ipahayag na ang buong natitirang Pautang ay agarang due at babayaran.}
\item Enforce rights against the collateral.
\item \textit{Ipataw ang mga karapatan laban sa kolateral.}
\item Impose penalty interest at the rate of 15\% per month on overdue amounts.
\item \textit{Magpataw ng parusang interes sa rate na 15\% kada buwan sa mga overdue na halaga.}
\end{itemize}
\section*{Article VIII – Penalties and Fees}
8.1 Late payments shall incur a penalty of increase interest. \\
\textit{(8.1 Ang mga late na bayarin ay magkakaroon ng parusa na increase interest.)}
8.2 All expenses incurred in enforcing this Agreement, including attorney's fees and court costs, shall be charged to the Borrower. \\
\textit{(8.2 Lahat ng gastusin sa pagpapatupad ng Kasunduang ito, kabilang ang bayarin sa abogado at gastos sa korte, ay sisingilin sa Nanghihiram.)}
\section*{Article IX – Governing Law and Dispute Resolution}
9.1 This Agreement shall be governed by and construed in accordance with the laws of the Republic of the Philippines. \\
\textit{(9.1 Ang Kasunduang ito ay pamamahalaan at ipapaliwanag alinsunod sa mga batas ng Republika ng Pilipinas.)}
9.2 Any dispute shall be resolved through amicable settlement, failing which it shall be submitted to arbitration or to the exclusive jurisdiction of the courts of the Philippines. \\
\textit{(9.2 Anumang hindi pagkakasundo ay lutasin sa pamamagitan ng maayos na pamamaraan, at kung hindi ito malutas, ito ay isusumite sa arbitrasyon o sa eksklusibong hurisdiksyon ng mga korte ng Pilipinas.)}
\section*{Article X – Miscellaneous}
10.1 Notices – All notices shall be in writing and delivered personally, by courier, or registered mail to the addresses provided herein. \\
\textit{(10.1 Mga Abiso – Lahat ng abiso ay dapat nakasulat at ihahatid nang personal, sa pamamagitan ng courier, o rehistradong koreo sa mga address na ibinigay dito.)}
10.2 Severability – If any provision is held invalid, the remainder shall continue in effect. \\
\textit{(10.2 Separabilidad – Kung ang anumang probisyon ay itinuring na hindi wasto, ang natitira ay mananatiling may bisa.)}
10.3 Amendments – Any modification must be in writing and signed by both Parties. \\
\textit{(10.3 Mga Pagbabago – Anumang pagbabago ay dapat nakasulat at nilagdaan ng parehong Panig.)}
10.4 Entire Agreement – This Agreement embodies the entire understanding between the Parties. \\
\textit{(10.4 Buong Kasunduan – Ang Kasunduang ito ay naglalaman ng buong pag-unawa sa pagitan ng mga Panig.)}
\section*{Signatures / Lagda}
IN WITNESS WHEREOF, the Parties have hereunto affixed their signatures on the date and place first above written. \\
\textit{(Bilang patunay, ang mga Panig ay lumagda sa kasunduang ito sa petsa at lugar na nakasaad sa itaas.)}
\vspace{0.5cm}
\begin{tabular}{p{0.45\textwidth} p{0.45\textwidth}}
\textbf{Lender / Tagapagpautang} & \textbf{Borrower / Nanghihiram} \\
\vspace{2cm} \hrulefill & \vspace{2cm} \hrulefill \\
Signature / Lagda & Signature / Lagda \\
Name / Pangalan: \underline{[Lender Name]} & Name / Pangalan: \underline{[Borrower Name]} \\
Designation / Katungkulan: \underline{[Lender Designation]} & Designation / Katungkulan: \underline{[Borrower Designation]} \\
Date / Petsa: \underline{September 17, 2025} & Date / Petsa: \underline{September 17, 2025} \\
\end{tabular}
\vspace{0.5cm}
\begin{tabular}{p{0.45\textwidth} p{0.45\textwidth}}
\textbf{Witness / Saksi} & \textbf{Witness / Saksi} \\
\vspace{2cm} \hrulefill & \vspace{2cm} \hrulefill \\
Signature / Lagda & Signature / Lagda \\
Name / Pangalan: \underline{[Witness 1 Name]} & Name / Pangalan: \underline{[Witness 2 Name]} \\
\end{tabular}
\newpage
\section*{Acknowledgment / Pagpapatibay}
Republic of the Philippines / Republika ng Pilipinas ) \\
Province of [Province] / Lalawigan ng [Province] ) S.S. \\
City/Municipality of [City] / Lungsod/Bayan ng [City] )
\vspace{0.5cm}
BEFORE ME, a Notary Public for and in the City/Municipality of [City], this 17th day of September 2025, personally appeared: \\
\textit{(SA HARAP KO, isang Notary Public sa Lungsod/Bayan ng [City], ngayong ika-17 ng September 2025, ay personal na humarap:)}
\vspace{0.3cm}
Name / Pangalan: \underline{[Lender Name]} \\
Competent Evidence of Identity / Katibayan ng Pagkakakilanlan: \underline{[Lender ID]} \\
Issued on / Inisyu noong: \underline{[Issue Date]} \\
Issued at / Inisyu sa: \underline{[Issue Place]} \\
\vspace{0.3cm}
Name / Pangalan: \underline{[Borrower Name]} \\
Competent Evidence of Identity / Katibayan ng Pagkakakilanlan: \underline{[Borrower ID]} \\
Issued on / Inisyu noong: \underline{[Issue Date]} \\
Issued at / Inisyu sa: \underline{[Issue Place]} \\
\vspace{0.5cm}
Known to me and to me known to be the same persons who executed the foregoing Loan Agreement consisting of \pageref{LastPage} pages, including this page where this acknowledgment is written, and they acknowledged to me that the same is their free and voluntary act and deed. \\
\textit{(Na aking nakilala at kinilala na sila rin ang mga taong lumagda sa nasabing Kasunduan sa Pautang na binubuo ng \pageref{LastPage} pahina, kabilang ang pahinang ito kung saan nakasulat ang pagpapatibay na ito, at kinilala nila sa akin na ito ay kanilang malaya at kusang-loob na gawa.)}
\vspace{0.5cm}
IN WITNESS WHEREOF, I have hereunto set my hand and affixed my notarial seal on the day, year, and place first above written. \\
\textit{(Bilang patunay, ako'y lumagda at naglagay ng aking notarial seal sa petsa, taon, at lugar na una nang binanggit.)}
\vspace{0.5cm}
\textbf{Notary Public / Notaryo Publiko} \\
\vspace{2cm} \hrulefill \\
Name / Pangalan: \underline{[Notary Name]} \\
PTR No. \underline{[PTR No.]} / Date / Petsa \underline{[Date]} / Place / Lugar \underline{[Place]} \\
IBP No. \underline{[IBP No.]} / Date / Petsa \underline{[Date]} / Place / Lugar \underline{[Place]} \\
Roll No. \underline{[Roll No.]} \\
Commission No. \underline{[Commission No.]} \\
Until / Hanggang: \underline{[Until Date]} \\
\label{LastPage}
\end{document}
